\documentclass[11 pt]{article}

\usepackage{graphics}
\usepackage{hyperref}

\renewcommand{\familydefault}{\sfdefault}

\setlength{\textwidth} {6.5 true in}
\setlength{\textheight}{9.5 true in}
\setlength{\hoffset}   {-0.5 true in}
\setlength{\voffset}   {-1.0 true in}
\pagestyle{empty}

\begin{document}

\section*{PHYS328W: Guidelines for Log Books}

Your audience is your future self (and in this course, also me, since
I am evaluating your log book). Your notes are clear and detailed
enough that when you read them in a few weeks, as you're writing a
paper on your work, you (and I) will be able to figure out what you
did.

\subsection*{Experimental Work}
\begin{enumerate}
  \item Use the available equipment effectively.

  \item Collect enough data of good enough quality to extract credible
    results.

  \item If things go wrong, make a reasonable effort to correct
    the problems.
\end{enumerate}
  
\subsection*{Log Book Content}
\subsubsection*{Experiment}
\begin{enumerate}
\item Entries include date and time and who was present.

\item Describe the configuration of the apparatus is described in enough 
  detail that a reader could reproduce comparable raw data.

\item Record any dimensions, settings, and other configuration
  parameters that might be needed to make sense of the raw data.

\item Include clearly-labeled circuit diagrams.

\item Record raw data in the log book if at all practical.

\item Include the units of all physical quantities.
\end{enumerate}

\subsubsection*{Analysis}
\begin{enumerate}
\item Perform your entire analysis in the log book.
  
\item Describe your methods of analysis in enough detail that a reader
  could perform the same analysis of your raw data and obtain the same
  results.

\item Include images of derivations of important equations.
\end{enumerate}

\end{document}
