\documentclass[11 pt]{article}

\usepackage{graphics}
\usepackage{hyperref}

\renewcommand{\familydefault}{\sfdefault}

\setlength{\textwidth} {6.5 true in}
\setlength{\textheight}{9.5 true in}
\setlength{\hoffset}   {-0.5 true in}
\setlength{\voffset}   {-1.0 true in}
\pagestyle{empty}

\begin{document}

\section*{PHYS328W: Guidelines for Papers}

\subsection*{Writing}
\begin{enumerate}
\item Describe your work in enough detail that others can attempt to
  reproduce your results with similar equipment. Describe what you
  did. Avoid writing in the style of an instruction
  manual. Avoid lists of equipment.  

\item Use clear, precise, objective language. Replace or supplement
  ambiguous language like ``acceptable'' or ``pretty good,'' with
  language that describes what you mean.

\item Effectively integrate equations and references to figures and
  tables into your writing. They are part of the story. 

\item Organize your sentences and paragraphs in a coherent, logical
  sequence. 

\item Engage fully in the process of writing and revising drafts.

\end{enumerate}

\subsection*{Analysis and Reasoning}
\begin{enumerate}
\item Report a \textit{complete} analysis of your data. If more
  information can be extracted from your data, extract it.

\item Address uncertainties as indicated in the lab
  assignment.

  \emph{We do not emphasize uncertainties very much in the
    electronics context, where agreement between measurement,
    theory, and simulation is generally quite good, and much of the
    time, devices are designed to allow for operating parameters to
    vary within acceptable ranges. This is not an invitation to
    develop the habit of ignoring uncertainties in general!}

\item Commentary
  \begin{enumerate}
  \item Comment on the agreement of your results with
    theory/established results.

  \item Describe how one might \textit{productively} improve on your work.

  \item Only make claims supported and accompanied by compelling evidence.

  \end{enumerate}

\end{enumerate}

\subsection*{Mechanics}
\begin{enumerate}
\item Figures and Tables
  \begin{enumerate}
  \item Label the axes of all of your graphs. Include units.

  \item Include captions with all figures and tables, and refer to
    each figure and table at least once in the text of your
    paper. Use the \LaTeX\ \verb+\label{}+ and \verb+\ref{}+
    commands to manage figure and table references.

  \end{enumerate}

\item References 
  \begin{enumerate}
  \item Bibliography format: Table I of 
    \url{http://forms.aps.org/author/styleguide.pdf}.

  \item Use the \LaTeX\ \verb+thebibliography+ environment and
    \verb+\cite{}+ commands to produce numbered citations. List
    references in the order of first citation in your paper.
  \end{enumerate}

\item Include units and uncertainties with all measurements and
  results derived from them.

\item Find and fix grammar, punctuation, spelling, and typesetting
  errors.

\end{enumerate}

\end{document}
